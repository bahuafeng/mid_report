% !Mode:: "TeX:UTF-8"
\section{后期拟完成的研究工作及进度安排}

在仿真系统开发出来之后,针对软件层服务不确定性决策,还缺少三个实验:

实验1:同一服务方案下,执行过程中面对一组按时间先后次序发生的不确定性事件,分别采用~MDP~和贪心策略,对比决策效果。

实验2:针对同一服务方案,在服务失败赔偿参数~($fc$)~取不同值的情况下,对决策动作和相应收益的影响。

实验3:针对不同结构、不同规模的流程,针对一次决策所耗费时间的对比。

针对业务层面不确定性决策,还缺少进行三个实验:

实验1:多用户同时请求,执行过程中随机发生用户需求和资源变化的不确定性事件,仿真服务执行结果;

实验2:用户需求发生变化,资源不发生变化,仿真收益随需求的变化趋势;

实验3:资源发生变化,需求不发生变化,仿真收益随资源的变化趋势。

针对这几个实验,分别编写测试程序进行测试,验证算法可行性和有效性。